%%%%%%%%%%%%%%%%%%%%%%%%%%%%%%%%%%%%%%%%%%%%%%%%%%%%%%%%%%%%%%%%%%%%%%%%%%%%%%%%
%2345678901234567890123456789012345678901234567890123456789012345678901234567890
%        1         2         3         4         5         6         7         8

\documentclass[10 pt]{article}  % Comment this line out if you need a4paper

%\documentclass[a4paper, 10pt, conference]{ieeeconf}      % Use this line for a4 paper
\bibliographystyle{IEEEtran} 

\usepackage{amssymb}
\usepackage{amsmath}
\usepackage{commath}
\usepackage{mathtools}
\DeclarePairedDelimiter{\ceil}{\lceil}{\rceil}
\usepackage{bbm}
\usepackage{hyperref}
\usepackage{import}
\usepackage{paralist}
\usepackage[pdf]{svg}
\usepackage{IEEEtrantools}

% See the \addtolength command later in the file to balance the column lengths
% on the last page of the document

% The following packages can be found on http:\\www.ctan.org
\usepackage{graphics} % for pdf, bitmapped graphics files
\usepackage{epsfig} % for postscript graphics files
%\usepackage{mathptmx} % assumes new font selection scheme installed
%\usepackage{times} % assumes new font selection scheme installed
%\usepackage{amsmath} % assumes amsmath package installed
%\usepackage{amssymb}  % assumes amsmath package installed

\usepackage{todonotes}
\usepackage[normalem]{ulem}
\date{7 February 2017}
\title{\LARGE \bf
Simple Inverse Kinematics Using Ipopt}

\author{Stefano Dafarra}



%math commands
\DeclareMathOperator*{\minimize}{minimize}
\DeclareMathOperator*{\argmin}{argmin}

\begin{document}
	
\maketitle
\thispagestyle{empty}
\pagestyle{empty}


%%%%%%%%%%%%%%%%%%%%%%%%%%%%%%%%%%%%%%%%%%%%%%%%%%%%%%%%%%%%%%%%%%%%%%%%%%%%%%%%
%\begin{abstract}
%    Balancing and reacting to strong and unexpected pushes is a critical requirement for humanoid robots. 
%    The Capture Point approach is a criterion successfully used in literature to balancing and making biped robot walks.
%    We recently designed a capture point based approach which interfaces with momentum-based torque controllers, and we implemented and validated it on the iCub humanoid robot.
%    In this work we implement a Receding Horizon control, also known as Model Predictive Control, to add the possibility to predict the future evolution of the robot, especially the constraint switching given by the hybrid nature of the system.
%    We prove that the proposed MPC extension makes the step-recovery controller more robust and reliable when executing the recovery strategy.
%    Experiments in simulation show the results of the proposed approach.
%\end{abstract}

%%%%%%%%%%%%%%%%%%%%%%%%%%%%%%%%%%%%%%%%%%%%%%%%%%%%%%%%%%%%%%%%%%%%%%%%%%%%%%%%

\import{tex/}{problem}
\import{tex/}{jacobian}
\import{tex/}{twistMap}
\import{tex/}{constraints}

\addtolength{\textheight}{0cm}   % This command serves to balance the column lengths
                                  % on the last page of the document manually. It shortens
                                  % the textheight of the last page by a suitable amount.
                                  % This command does not take effect until the next page
                                  % so it should come on the page before the last. Make
                                  % sure that you do not shorten the textheight too much.


\addcontentsline{toc}{section}{References}

\bibliography{IEEEabrv,Bibliography}

\end{document}
